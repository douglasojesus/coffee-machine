\chapter{RESULTADOS}
Por fim, a MEF realizada apresentou resultados positivos em todos os testes realizados. Entretanto, isso não resume todo o projeto. No início houve dificuldade na escolha do tipo de máquina, em qual seria mais ideal para a implementação do problema proposto. Não obstante, com reforço no referencial teórico e na percepeção da necessidade das saídas em lidarem com as entradas, foi definido o tipo de Mealy. Correções foram feitas em todo o processo e isso fez com que a construção tenha sido contínua, o que definiu um resultado mais embasado e fortificado em todos os seus processos. Visualizando o RTL Viewer, é possível verificar que o sistema segue em conformidade com o desenho do circuito realizado. A compilação bem sucedida também garante a falta de problemas gramaticais e léxicos do código, possibilitando a implementação física na placa de código.