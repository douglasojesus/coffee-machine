\chapter{INTRODUÇÃO}

A indústria do café é um dos setores mais importantes e lucrativos do mercado de bebidas. Com a crescente demanda por café em todo o mundo e seguindo o problema fictício proposto pela disciplina do Módulo Integrador de Circuitos Digitais com base na metodologia PBL, a empresa MANDACARU SA identificou uma oportunidade de negócio e decidiu investir no desenvolvimento de Vending Machines para atender aos amantes da bebida durante seus "coffee breaks" em ambientes de trabalho.

Nesse contexto, a MANDACARU SA já desenvolveu um protótipo de hardware para a Coffee Machine, que consiste em um compartimento de água, um compartimento para as cápsulas, um compartimento para copos, botões de seleção dos produtos, um dispensador, uma bomba de pressurização e um circuito elétrico de aquecimento por indução. No entanto, para automatizar o processo de vendas, é necessário implementar um circuito digital capaz de reconhecer e validar as moedas inseridas pelos usuários.

O objetivo deste relatório técnico é apresentar uma proposta de construção lógica dos circuitos internos da Coffee Machine, utilizando máquinas de estados, a fim de garantir o correto funcionamento do sistema de vendas. Serão abordadas as principais características do protótipo, como os sensores de nível mínimo no compartimento de água e os sensores de presença nos compartimentos de cápsulas e de recepção de copos. Além disso, serão discutidos os requisitos específicos que o circuito digital deve atender, como a validação das moedas de acordo com os preços dos produtos, a exibição de mensagens de erro e a correta sequência de acionamento dos circuitos internos.

Para atingir esse objetivo, serão realizadas atividades como a análise detalhada dos requisitos do sistema, o projeto da lógica de funcionamento utilizando máquinas de estados, a seleção dos componentes eletrônicos adequados e a implementação do circuito digital. Serão consideradas também as diretrizes estabelecidas, como a exibição de informações relevantes no display da máquina, a utilização de LEDs para representar as saídas do sistema e a correta identificação e tratamento de situações de erro.

Ao final deste relatório, espera-se apresentar uma solução técnica eficiente e funcional para automatizar o processo de vendas da Coffee Machine, garantindo uma experiência satisfatória para os usuários e contribuindo para o sucesso do empreendimento da MANDACARU SA no mercado de café.

O projeto em questão requer a implementação de um circuito digital para a Coffee Machine, a fim de automatizar o processo de vendas através do reconhecimento de moedas. Dessa forma, será possível garantir a correta seleção e pagamento dos produtos disponíveis na máquina.

As funcionalidades essenciais que o circuito digital deve oferecer incluem a validação das moedas inseridas, verificando se correspondem aos valores dos produtos (café expresso, café com leite, chá de camomila e cappuccino), exibindo mensagens de erro caso seja detectada uma moeda inválida e devolvendo as cédulas ou moedas ao cliente. Além disso, o circuito deve monitorar os níveis mínimos de água, a presença de cápsulas e a disponibilidade de copos, exibindo mensagens de erro apropriadas caso algum desses elementos esteja em falta.

Para proporcionar uma interação clara e intuitiva com o usuário, a Coffee Machine deverá contar com um display que informe o produto escolhido, o valor total das cédulas inseridas e eventuais códigos de erro associados a ações incorretas. Os resultados das ações realizadas pelo circuito interno também serão representados por LEDs, que servirão como indicadores visuais das saídas do sistema.

É importante destacar que a máquina deve permanecer funcional mesmo em caso de situações adversas, evitando travamentos durante o processo de venda. Para isso, será necessário desenvolver uma lógica eficiente que permita a correta sequência de acionamento dos circuitos internos, ativando o aquecimento por indução 2 segundos após o início da pressurização de água, possibilitando a extração adequada do café e sua posterior disponibilização ao cliente pelo dispensador.
